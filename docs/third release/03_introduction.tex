\pagenumbering{arabic}
\setcounter{page}{1}

\chapter{Einleitung}\label{ch:einleitung}

In dem zweiten Release sind verschiedene Funktionalitäten der Spielwelt, die den Nutzern die Möglichkeit bieten, ihre Trainer freiräumlich zu bewegen und mit ihren Trainern verschiedene Spielareale entdecken zu können, implementiert worden. Außerdem ist die Funktion des Mehrspieler-Modus bereits integriert, in dem der Nutzer mit anderen Trainern spielen kann. In diesem Release werden diese Funktionalitäten weiter ausgebaut, sodass der Nutzer interaktiv mit anderen Trainern zusammenspielen kann. Der Nutzer hat somit die Möglichkeit durch Dialoge mit anderen \Gls{NPC}-Trainern zu interagieren, Starter-Monster zu erhalten, Kämpfe mit anderen (NPC-)Trainern oder wilden Monstern zu starten und zu führen sowie eigene Monster zu verwalten und für sie zu sorgen. Darüber hinaus ist es mit dem Kunden vereinbart worden, drei Bonusfeatures für das Spiel zu entwickeln. Als erstes Bonusfeature werden in dem Spielbildschirm Audio-Titel abgespielt, um das Spielerlebnis heiter zu gestalten.
Zusätzlich ist das Spiel einsteigerfreundlich, da ein futuristisches Handy als Hilfestellung zur aktuellen Spielsituation implementiert ist.
Außerdem kann der Nutzer eigene Einstellungen bzgl. der Tastenkürzel treffen, um mit den bevorzugten Tasten für die Bewegung und Interaktion zu spielen.
Als Letztes werden die Trainereinstellungen erweitert, sodass der Nutzer seinen Trainer nicht nur löschen, sondern auch seinen Namen ändern und einen anderen Charakter auswählen kann. 

Im Zuge dieser Dokumentation werden die Anforderungen des dritten Releases anhand der Mockups und deren Implementierung vorgestellt. Dabei werden die Unterschiede zwischen den Mockups und der tatsächlichen Implementierung verdeutlicht.
Des Weiteren werden die technischen Grundlagen dieses Releases beschrieben und abschließend werden die User-Storys, Tasks und eventuelle Bugs bewertet.